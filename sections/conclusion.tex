\section{Conclusion}
\begin{frame}{Conclusion}

% 	\begin{minipage}[c][6.5cm]{\textwidth}
		\begin{itemize}\itemfill
			\item strongly improved fabrication of 3D diamonds
			\begin{itemize}
				\item \SI{40}{\times} more cells
				\item smaller cell size down to \SI{50x50}{\micro\meter}
				\item thinner columns down to \SI{2}{\micro\meter}\vspace*{2ex}
			\end{itemize}
			\item 3D Detectors work well in pCVD diamond
			\item general reasons for inefficiencies:
			\begin{itemize}
				\item no charge created in the volume of the electrodes (\SI{.4}{\%} for shown devices)
				\item region with low electric field
				\item missing/broken columns\vspace*{2ex}
			\end{itemize}
			\item \SI{99.2\pm .3}{\%} efficiency in \SI{3x2}{} ganged device
			\item \SI{98.2\pm .2}{\%} efficiency in \SI{5x1}{} ganged device
			\begin{itemize}
				\item most likely due to different processing\vspace*{2ex}
			\end{itemize}
			\item consistent mean charge measurements for all devices: \SI{\sim14500}{e} @ CERN SPS
			\item 3D has largest charge collection of all pCVD diamond detectors
			\begin{itemize}
				\item work towards quantifying the charge collection in both non- and irradiated devices
			\end{itemize}
		\end{itemize}
% 	\end{minipage}
	
\end{frame}
